% !TEX encoding = UTF-8 Unicode
%%%%%%%%%%%%%%%%%%%%%%%%%%%%%%%%%%%%%%%%%%
%% Proceedings of the Niccolò Cusano Scientific Conferences (NCSC)
%% LaTeX Template @ www.edicusano.it/LaTeX.html
%% Version 1.1 (27/05/23)
%%
%%
%% Original author: Pietro Oliva  --  e-mail: pietro.oliva@unicusano.it
%% The CusanoTwo class was created by Pietro Oliva 
%%%%%%%%%%%%%%%%%%%%%%%%%%%%%%%%%%%%%%%%%%
\documentclass{CusanoOne}
%%+++++++++++++++++++++++++++++++++++++++++
%% PACKAGES: 
%% THE COMMONLY USED ONES AND MUCH MORE
%% ARE ALREADY LOADED IN THE CLASS
%%+++++++++++++++++++++++++++++++++++++++++
\begin{document}

%%%%%%%%%%%%%%%%%%%%%%%%%%%!!!!!!
\date{}%%don't need date to appear%%%!!!!!!
\setcounter{Maxaffil}{0}%%%!!!!!!  === DON'T TOUCH ====
\renewcommand\Affilfont{\itshape\footnotesize}%%%!!!!!!
%%%%%%%%%%%%%%%%%%%%%%%%%%%!!!!!!


%%%+++++++++++++++++++++++++++++++++++++++++
%%	TITLE AND AUTHORS/AFFILIATION
%%%+++++++++++++++++++++++++++++++++++++++++

\title{A stunning title for your paper} % For titles, only capitalize the first letter

\author[a]{John Doe}



\affil[a]{Universit\`a degli Studi di Poggibonsi, Via qualcosa 22 – Rome, Italy}



\maketitle % The \maketitle command is necessary to build the title page

\begin{abstract}

\lipsum[1]%%dummy text from lipsum package

\end{abstract}


%%%%%%%%%%%%%%%
%%   The first section must %
%%    be "Introduction"        %
%%%%%%%%%%%%%%%

\introduction%%%DON'T TOUCH

The beginning;  example of citing \cite{PhysRevB.49.3030, doi:10.1063/1.1935039}.
Please use text sub and super scripts inline: \textsuperscript{90}Sr was used for A\textsubscript{one}.
Numbers in the text line: In 2007 we spent \num{1500000} dollars for the experiment involving 1230 antennas and the measurement was 12,27\footnote{don't write $12,27$: it will cause extraspace}.

\begin{equation}\label{eq:random}
\ddx F=\ddt G=\tan{34\degree} \qquad \intl{0}{\infty}{f(x)}{x} \quad \pfrac{g(x,y)}{x}
\end{equation}

Referencing equation \eqref{eq:random}. Aligned formulas:
\begin{equation}
\begin{split}
a & = b+c-d \\
& = e-f \\
& = g
\end{split}
\end{equation}
SI units must be used both in math mode, $r=\SI{0.8768(11)e-15}{m}$, and in text: ``the Gas constant is expressed in \si{\joule\per\mole\per\kelvin}''. 
%%please refer to siunitx package documentation

\section{Results}\label{result}
There are almost no limits to what you can use:

Phonetic \textipa{ABCDEFGHIJKLMNOPQRSTUVWXYZ}

Greek \greco{>Akous'ilaon d`e Q'aos m`en <upot'ijesja'i moi dokei t`hn pr'wthn >arq`hn}

Cyrillic \textcyr{U jedna usta nosi vile i vetrovi, druga usta - izdat i zle bolesti}

Runes \textara{bs\tripledot wib\doublecross nk jo\th soInw\sfour m} \textarl{hjmrst}

Ornaments \adforn{26}\adforn{13}\adforn{3}\adforn{74}\adforn{62}\adforn{32} 
\adforn{21}\quad\adforn{11}\quad\adforn{49}


Olmec  \EOxiv \EOofficerII \EOtu \EOkij \EOeat \EOSprinkle \EOScorpius

Etruscan    \textetr{abgdefghilmnopqrstuvxywz}

Persian arcaic \textcopsn{\Occa \Ora \Oking\OAura\Oa\Oi\Ou\Oka}

Fenician: \textphnc{abgdefghilmnopqrstuvxywz}

Protosemitic \textproto{abgdefghilmn'opqrTtuvxSywz}

Sud-Arabia Arcaic: \textsarab{abgdefghilmn'opqrTtuvxSywz}

Various: \NoWash \NoBleech \IroningI \Industry \Info \Recycling \EUR  \CESign \Laserbeam \Biohazard \Radioactivity\Mercury \Venus \Mars

Hebrew \begin{cjhebrew}
b*:re’+siyt b*ArA’ ’E:lohiym ’et ha+s*
Amayim w:’et hA’ArE.s; w:hA’ArE.s
hAy:tAh tohU wAbohU w:.ho+sEk: ‘al--p*:ney t:hOm w:rU/a.h ’E:lohiym
m:ra.hEpEt ‘al--p*:ney ham*Ayim;
\end{cjhebrew}

Arabic \begin{RLtext}
fa-qAla lahu .sadIquhu: ’innI ’asma‘u .himAraka yA
^gu.hA yanhaqu.  fa-qAla lahu ^gu.hA: .garIbuN ’amruka yA .sadIqI!
’a-tu.saddiqu al-.himAra wa-tuka_d_dibunI?
\end{RLtext}

Moreover, if you really feel you have to insert a .jpg or a .png inline, you can do it like shown here: Tiamat (\inschar[height=2ex]{tiamat}) can be considered one of the most primordial goddesses. 

\subsection{Simulations}

\bit
\item Hello.
\item Goodbye.
\eit
\ben
\item Hello.
\item Goodbye.
\een

\subsubsection{Further discussion}

Chemical can be included:
\begin{center}
\chemname{\chemfig{*6((-HO)-=%
-(-(<[::60]OH)-[::-60]-%
[::-60,,,2] HN-[::+60]CH_3)=%
-(-HO)=)}}{Adrenalin}
\end{center}


\subsubsection{Special remarks}

A tabular is used when text is prevailing:
\begin{center}
\begin{tabular}{lcr}
\toprule
Quantity & Symbol & Units \\
\midrule
force       & $F$     & newton \\
energy    & $E$     & joule \\
potential  & $V$     & volt \\
\bottomrule
\end{tabular}
\end{center}
while an array is preferred when math is important:
\[
\begin{array}{ll}
\toprule
f(x)   & f’(x) \\
\midrule
x^n    & nx^{n-1} \\
e^x    & e^x \\
\sin x & \cos x \\
\bottomrule
\end{array}
\]
%%%+++++++++++++++++++++++++++++++++++++++++
%%	ACKNOWLEDGEMENTS
%%%+++++++++++++++++++++++++++++++++++++++++

\section*{Acknowledgements}
This work (in particular section~\ref{intro}) was partially supported by a grant from Santa.



%%%+++++++++++++++++++++++++++++++++++++++++
%%	BIBLIOGRAPHY
%%%+++++++++++++++++++++++++++++++++++++++++
%% NCSC does not support submission of supporting .tex files such as BibTeX.
%% All references must be included in the article .tex document. 
%% If you use BibTeX, check the <filename>.bbl file into your
%% .tex folder and copy the reference listings
%% from your .bbl file using the following bibliography environment example.  
\begin{thebibliography}{99}

\bibitem{PhysRevB.49.3030}
W.~Wu, S.~Fahy, Molecular-dynamics study of single-atom radiation damage in
  diamond, Phys. Rev. B 49 (1994) 3030--3035.
\newblock \href {http://dx.doi.org/10.1103/PhysRevB.49.3030}
  {\path{doi:10.1103/PhysRevB.49.3030}}.

\bibitem{doi:10.1063/1.1935039}
A.~D. Sio, J.~Achard, A.~Tallaire, R.~S. Sussmann, A.~T. Collins, F.~Silva,
  E.~Pace, Electro-optical response of a single-crystal diamond ultraviolet
  photoconductor in transverse configuration, Applied Physics Letters 86~(21)
  (2005) 213504--6.
\newblock \href {http://dx.doi.org/10.1063/1.1935039}
  {\path{doi:10.1063/1.1935039}}.

\end{thebibliography}

\end{document}